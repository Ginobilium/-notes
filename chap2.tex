\ifx\allfiles\undefined
\documentclass{book}
\usepackage{ctex}
\usepackage{amsmath}
\usepackage{amssymb}
\usepackage[mathscr]{eucal}
\newcommand{\Eqv}{\Leftrightarrow}
\newcommand{\eqv}{\leftrightarrow}
\newcommand{\To}{\Rightarrow}
\newcommand{\A}{\forall}
\newcommand{\E}{\exists}
\newcommand{\no}{\varnothing}
\newcommand{\scr}[1]{\mathscr{#1}}
\newcommand{\nsubset}{\not\subset}
\begin{document}
\else
\chapter{集合的概念与运算}
\fi
\section{集合的概念}
\noindent
集合在朴素集合论中不能精确定义\\
一些对象的整体就构成集合\\
\subsection{属于}
$a\in A\quad a\notin A$\\
\subsection{罗素悖论}
\noindent
$X=\{x|x\ne \no\}, \{a\}\ne\no, \{a\}\in X, X\in X$\\
$\no\notin\no, \{a\}\notin \{a\}, \E X(X\notin X)$\\
$S=\{X|X\notin X\}, S\in S?$\\
集合论公理(ZFC):
\begin{enumerate}
    \item{外延公理} 所含元素相同的两个集合是相等的
    \item{空集存在公理} 空集合存在
    \item{无序对公理} 对任意集合$a,b$, $\{a,b\}$存在
    \item{并集公理} 对任意集合$\scr{A}, \cup \scr{A}$存在
    \item{幂集公理} 对任意集合$A, \scr{P}(A)$存在
    \item{联集公理}
\end{enumerate}
\subsection{多重集}
允许元素多次重复出现的集合(元素的重复度)
\subsection{特征函数法}
集合$A$的特征函数是$\chi_A(x)$:
$$
\chi_A(x)=
    \begin{cases}
        1, \mbox{若}x\in A;\\
        0, \mbox{若}x\notin A.
    \end{cases}
$$
对多重集,$\chi_A(x)=x$在$A$中的重复度
\section{集合之间的关系}
\subsection{子集}
\noindent
$B\subseteq A\Eqv \A x(x\in B\to x\in A)$\\
$B\nsubseteq A\Eqv \E x(x\in B\land x\notin A)\quad \mbox{即}(\neg(B\subseteq A))$
\subsection{相等}
$A=B\Eqv A\subseteq B\land B\subseteq A\Eqv \A x(x\in A\eqv x\in B)$
\subsection{$\subseteq$的性质}
\noindent
$A\subseteq A$\\
若$A\subseteq B$,且$A\neq B$,则$B\nsubseteq A$\\
若$A\subseteq B$,且$B\subseteq C$,则$A\subseteq C$
\subsection{真子集}
$A\subset B\Eqv A\subseteq B\land A\neq B$
\subsection{$\subset$的性质}
\noindent
$A\nsubset A$\\
若$A\subset B$,则$B\nsubset A$\\
若$A\subset B, B\subset C$,则$A\subset C$
\subsection{空集}
$\no$\\
定理一:$\no\subseteq A\Eqv \A x(x\in\no\to x\in A)\Eqv\A x(0\to x\in A)\Eqv 1$\\
推论:空集唯一\\
设$\no_1$与$\no_2$都是空集,则\\
$\no_1\subseteq\no_2\land\no_2\subseteq\no_1\Eqv\no_1=\no_2$
\subsection{全集}
全集:如果限定所讨论的集合都是某个集合的子集,则称这个集合是全集,记作$E$
\subsection{幂集}
$A$的全体子集组成的集合$\scr{P}(A)$\\
$\scr{P}(A)=\{X\mid X\subseteq A\}$\\
$x\in\scr{P}(A)\Eqv x\subseteq A$
\subsection{$n$元集}
$A$是$n$元集$\Eqv x\subseteq A$
\subsection{集族}
由集合构成的集合
\subsection{指标集}
设$\scr{A}$是集族,若$\scr{A}=\{A_\alpha\mid\alpha\in S\}$则称$S$为$\scr{A}$的指标集\\
也记作$\scr{A}=\{A_\alpha\mid\alpha\in S\}=\{A_\alpha\}_{\alpha\in S}
\section{集合的运算}
\subsection{并集}
\noindent
并集:\\
$A\cup B=\{x|(x\in A)\lor(x\in B)\}$\\
$x\in A\cup B\Eqv (x\in A)\lor(x\in B)$\\
初级并:\\
$A_1\cup A_2\cup\dots\cup A_n=\{x\mid\E i(1\le i\le n\lor x\in A_i)\}$
\begin{align*}
\bigcup_{i=1}^{n} A_i=A_1\cup A_2\cup\dots\cup A_n\\
\bigcup_{i=1}^{\infty}=A_1\cup A_2\cup\dots
\end{align*}
\subsection{交集}
\subsection{不相交}
$A\cap B=\no$\\
互不相交
\subsection{相对补集}
$A-B=\{x\mid (x\in A)\land (x\notin B)\}$
\subsection{对称差}
$A\oplus B=\{x\mid (x\in A\land x\notin B)\lor (x\notin A\land x\in B)\}$\\
$A\oplus B=(A-B)\cup (B-A)=(A\cup B)-(A\cap B)$
\subsection{绝对补}
$\sim A=E-A$
\subsection{广义并}
设$\scr{A}$是集族,其所有集合的元素的全体,称为$\scr{A}$的广义并,记作$\cup\scr{A}$\\
$\cup\scr{A}=\{x\mid \E z(x\in z\land z\in\scr{A})\}=\cup_{\alpha\in S}A_\alpha$
\subsection{广义交}
$\cup\scr{A}=\{x\mid \A z(x\in z\to z\in\scr{A})\}$
\subsection{集合运算的优先级}
\noindent
第一级:补$\sim$,幂$\scr{P}$\\
第二级:广义并$\cup$,广义交$\cap$\\
第三级:并$\cup$,交$\cap$,相对补$-$,对称差$\oplus$
\section{容斥原理}

\ifx\allfiles\undefined
\end{document}
\fi

